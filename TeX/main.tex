% Document class.
\documentclass{article}
% Input encoding
\usepackage[utf8]{inputenc}
\usepackage[T1]{fontenc}
% Graphics.
\usepackage{graphicx}
\usepackage{float}
% Maths
\usepackage{amsmath}
% Spacing
\usepackage{xspace}

% Title and authors.
\title{FMNN05 \\\large{Report}}
\author{
    Stefan Eng \texttt{<atn08sen@student.lu.se>} \\
    <name> \texttt{<@student.lu.se>}
}

% Source code text related.
\usepackage{listings}
\usepackage{color}
\definecolor{dkgreen}{rgb}{0,0.6,0}
\definecolor{gray}{rgb}{0.5,0.5,0.5}
\definecolor{mauve}{rgb}{0.58,0,0.82}

\lstset{frame=tb,
  language=python,
  aboveskip=3mm,
  belowskip=3mm,
  showstringspaces=false,
  columns=flexible,
  basicstyle={\small\ttfamily},
  numbers=none,
  numberstyle=\tiny\color{gray},
  keywordstyle=\color{blue},
  commentstyle=\color{dkgreen},
  stringstyle=\color{mauve},
  breaklines=true,
  breakatwhitespace=true,
  tabsize=2
}

\begin{document}

\maketitle
% No page number on first side.
\thispagestyle{empty}

\newcommand{\todo}[1] {\textcolor{red}{\textbf{[TODO]: #1}\xspace}}
\newcommand{\bdf}[1] {\texttt{BDF-#1\xspace}}
\newcommand{\EE} {\texttt{Explicit Euler\xspace}}

\newpage

\section{Project 01}

    \subsection{Task 1}

      Task 1 requires a equation that describes the motion of an elastic
      pendulum where the bar is replaced with a spring. The spring is given
      the string constant $k$ with other attributes set to 1.
      \begin{flalign}
          & & & \dot{y_1} = y_3 & \\
          & & & \dot{y_2} = y_4 & \\
          & & & \dot{y_3} = -y_1\lambda{}(y_1, y_2) & \\
          & & & \dot{y_4} = -y_2\lambda{}(y_1, y_2) -1 &
      \end{flalign}

      where $\lambda(y_1, y_2) =
          k \frac
              {\sqrt{y_1^2+y_2^2}-1}
              {\sqrt{y_1^2+y_2^2}
            }$.

      The equation was implemented as follows:

      % Include code for task 1.
      \noindent
      \input{includes/code/task1.tex}

      With the resulting plot:

      % Include graphs for task 1.
      \input{includes/figures/task_1.tex}

    \subsection{Task 2}

      Task 2 consists of using Assimulo to implementing the \bdf{3} and
      \bdf{4} algorithms.

      % Include code for task2 BDF3.
      \input{includes/code/task2_BDF3.tex}

      \

      \noindent
      The source shown above is the Python implementation of the \bdf{3}
      algorithm used for the experiments that follow. The implementation makes
      use of the \texttt{fsolve} method from the \texttt{scipy} package in
      order to use Newton's method as its iteration method.

      The alpha values are precalculated values taken from \todo{Find
      source of our alpha values and insert here}. \\

      \noindent
      \bdf{4} is implemented in the same manner, the Python source for
      that implementation is shown below.

      \newpage

      % Include code for task2 BDF4.
      \input{includes/code/task2_BDF4.tex}

    \newpage
    \subsection{Task 3}

      Task 3 consists of evaluating what influence the variable $k$ (the
      spring stiffness) has on the choice of method used to simulate the
      pendulum. This was done by comparing the methods: \EE{}, \bdf{2}, \bdf{3}
      and \bdf{4} with the k-values 0.1, 1, 5, 10, 100 and 1000.

      \input{includes/figures/task3_ord1.tex}
      \input{includes/figures/task3_ord2.tex}
      \input{includes/figures/task3_ord3.tex}
      \input{includes/figures/task3_ord4.tex}

      \todo{Task 3 conclusions.}

    \newpage
    \subsection{Task 4}

      Task 4 requires that the tests from Task 3 are repeated, but
      instead of using the Python implementations of the different functions,
      the simulations should be done using CVODE instead.

      \input{includes/figures/CVODE_var_k.tex}

      \noindent
      When using CVODE, there are some values that can be changed to make the
      solver behave differently. These values are; ATOL, RTOL, MAXORD and
      discretization, the second part of the task involves changing these
      values and see how it affects the output and performance of the
      simulation.


      \subsubsection{Varying the ATOL-value}

        By using CVODE and running the simulation for the k-values: 1, 10 and 100
        in combination with the ATOL parameters 0.01, 0.55 and 1.0 the following
        plots and performance statistics were received.

        \input{includes/figures/CVODE_atol_0.tex}
        \input{includes/figures/CVODE_atol_1.tex}
        \input{includes/figures/CVODE_atol_2.tex}
%        \input{includes/figures/CVODE_atol_3.tex}

        \input{includes/figures/CVODE_atol_stats.tex}

      \newpage
      \subsubsection{Varying the RTOL-value}

        Using the same

        \input{includes/figures/CVODE_rtol_0.tex}
        \input{includes/figures/CVODE_rtol_1.tex}
        \input{includes/figures/CVODE_rtol_2.tex}
%        \input{includes/figures/CVODE_rtol_3.tex}

        \input{includes/figures/CVODE_rtol_stats.tex}

      \subsubsection{maxorder}

        \input{includes/figures/CVODE_maxorder_0.tex}
        \input{includes/figures/CVODE_maxorder_1.tex}
        \input{includes/figures/CVODE_maxorder_2.tex}
        \input{includes/figures/CVODE_maxorder_stats.tex}

      \subsubsection{discretization}

        \input{includes/figures/CVODE_discretization_0.tex}
        \input{includes/figures/CVODE_discretization_1.tex}
        \input{includes/figures/CVODE_discretization_stats.tex}

    \subsection{Task 5}
    \subsection{Task 6}

%    \input{includes/figures/var_ord_k_1000.tex}
%    \input{includes/figures/ord_4_var_k.tex}
%    \input{includes/figures/excited_pend_var_init.tex}

\end{document}
