% Document class.
\documentclass{article}
% Input encoding
\usepackage[utf8]{inputenc}
\usepackage[T1]{fontenc}
% Graphics.
\usepackage{graphicx}
\usepackage{float}
% Maths
\usepackage{amsmath}

% Title and authors.
\title{FMNN05 \\\large{Report}}
\author{
    Stefan Eng \texttt{<atn08sen@student.lu.se>} \\
    <name> \texttt{<@student.lu.se>}
}

% Source code text related.
\usepackage{listings}
\usepackage{color}
\definecolor{dkgreen}{rgb}{0,0.6,0}
\definecolor{gray}{rgb}{0.5,0.5,0.5}
\definecolor{mauve}{rgb}{0.58,0,0.82}

\lstset{frame=tb,
  language=python,
  aboveskip=3mm,
  belowskip=3mm,
  showstringspaces=false,
  columns=flexible,
  basicstyle={\small\ttfamily},
  numbers=none,
  numberstyle=\tiny\color{gray},
  keywordstyle=\color{blue},
  commentstyle=\color{dkgreen},
  stringstyle=\color{mauve},
  breaklines=true,
  breakatwhitespace=true,
  tabsize=2
}

\begin{document}

\maketitle
% No page number on first side.
\thispagestyle{empty}

\newpage

\section{Project 01}

    \subsection{Task 1}

%        Task 1 begins by defining the equations that describe the motion of an
%        elastic pendulum, where the bar is replaced with a spring with spring
%        constant $k$ and other attributes set to 1.

      Task 1 requires a equation that describes the motion of an elastic
      pendulum where the bar is replaced with a spring. The spring is given
      the string constant $k$ with other attributes set to 1.
      \begin{flalign}
          & & & \dot{y_1} = y_3 & \\
          & & & \dot{y_2} = y_4 & \\
          & & & \dot{y_3} = -y_1\lambda{}(y_1, y_2) & \\
          & & & \dot{y_4} = -y_2\lambda{}(y_1, y_2) -1 &
      \end{flalign}

      where $\lambda(y_1, y_2) =
          k \frac
              {\sqrt{y_1^2+y_2^2}-1}
              {\sqrt{y_1^2+y_2^2}
            }$.

      The equation was implemented as follows:

      % Include code for task 1.
      \noindent
      \input{includes/code/task1.tex}

      With the resulting plot:

      % Include graphs for task 1.
      \input{includes/figures/task_1.txt}

    \subsection{Task 2}

      Task 2 consists of using Assimulo to implementing the \texttt{BDF-3} and
      \texttt{BDF-4} algorithms.

      % Include code for task2 BDF3.
      \input{includes/code/task2_BDF3.tex}

      % Include code for task2 BDF4.
      \input{includes/code/task2_BDF4.tex}

    \subsection{Task 3}

      \input{includes/figures/task3_ord1.txt}
      \input{includes/figures/task3_ord2.txt}
      \input{includes/figures/task3_ord3.txt}
      \input{includes/figures/task3_ord4.txt}
    \subsection{Task 4}
      \input{includes/figures/CVODE_var_k.txt}

      \subsubsection{atol-values}

        \input{includes/figures/CVODE_atol_0.txt}
        \input{includes/figures/CVODE_atol_1.txt}
        \input{includes/figures/CVODE_atol_2.txt}
        \input{includes/figures/CVODE_atol_3.txt}

      \subsubsection{rtol-values}

        \input{includes/figures/CVODE_rtol_0.txt}
        \input{includes/figures/CVODE_rtol_1.txt}
        \input{includes/figures/CVODE_rtol_2.txt}
        \input{includes/figures/CVODE_rtol_3.txt}

      \subsubsection{maxorder}

        \input{includes/figures/CVODE_maxorder_0.txt}
        \input{includes/figures/CVODE_maxorder_1.txt}
        \input{includes/figures/CVODE_maxorder_2.txt}

      \subsubsection{discretization}

        \input{includes/figures/CVODE_discretization_0.txt}
        \input{includes/figures/CVODE_discretization_1.txt}

    \subsection{Task 5}
    \subsection{Task 6}

    \input{includes/figures/var_ord_k_1000.txt}
    \input{includes/figures/ord_4_var_k.txt}
    \input{includes/figures/excited_pend_var_init.txt}

\end{document}
